\appendix

\section{Additional Results}

In this section we showcase further examples of models generated by
our system. In the table below, the left panel shows the base input
view $I_0$, the middle panel shows auxiliary views used for
photoconsistency calculations, and the right panel shows the MAP model
$\Model$ inferred by our system. In all our experiments we used two
image auxiliary images per base image, which were sampled one second
before and one second after the base image in the video sequence. For
further details of parameter settings see the main paper.

\newcommand{\ResultIm}[7]{\includegraphics[width=#1\textwidth]{figures_addendum/#2/#3_#4_frame#5_#6.#7}}
\newcommand{\ShowcaseIm}[4]{\ResultIm{0.23}{resized_full_results}{#1}{#2}{#3}{#4}{jpg}}
\newcommand{\AuxIm}[4]{\ResultIm{0.1}{aux_links/resized}{#1}{#2}{#3}{#4}{jpg}}
\newcommand{\ShowcaseRow}[3]{
        \ShowcaseIm{#1}{#2}{#3}{orig} &
        \AuxIm{#1}{#2}{#3}{aux0}
        \AuxIm{#1}{#2}{#3}{aux1} &
        \ShowcaseIm{#1}{#2}{#3}{dp} &
}
\newcommand{\ShowcaseRowLast}[3]{  % contains no final auxiliary image
        \ShowcaseIm{#1}{#2}{#3}{orig} &
        \AuxIm{#1}{#2}{#3}{aux0}
        \AuxIm{#1}{#2}{#3}{aux0} &
        \ShowcaseIm{#1}{#2}{#3}{dp} &
}
\newcommand{\ShowcaseRowFirst}[3]{  % contains no previous auxiliary image
        \ShowcaseIm{#1}{#2}{#3}{orig} &
        \AuxIm{#1}{#2}{#3}{aux1}
        \AuxIm{#1}{#2}{#3}{aux1} &
        \ShowcaseIm{#1}{#2}{#3}{dp} &
}

\newcommand\ColHeadings{\textbf{Input (base view)} & \textbf{(auxiliary views)} & \textbf{Output of our system}}


\begin{center}
\begin{tabular}{cccc}
  \ColHeadings \\
  \ShowcaseRow{exeter}{bursary}{006} \\
  \ShowcaseRow{exeter}{bursary}{026} \\
  \ShowcaseRow{exeter}{bursary}{028} \\
  \ShowcaseRow{exeter}{mcr1}{012} \\
\end{tabular}

%\clearpage
\begin{tabular}{cccc}
  \ColHeadings \\
  \ShowcaseRow{exeter}{mcr1}{014} \\
  \ShowcaseRow{exeter}{mcr1}{015} \\
  \ShowcaseRow{exeter}{mcr1}{019} \\
  \ShowcaseRow{exeter}{mcr1}{025} \\
  \ShowcaseRow{exeter}{mcr1}{044} \\
  \ShowcaseRow{exeter}{mcr1}{051} \\
  \ShowcaseRowLast{exeter}{mcr1}{054} \\
\end{tabular}

\clearpage
\begin{tabular}{cccc}
  \ColHeadings \\
  \ShowcaseRow{lab}{atrium2}{009} \\
  \ShowcaseRow{lab}{foyer1}{008} \\
  \ShowcaseRow{lab}{foyer1}{015} \\
  \ShowcaseRow{lab}{foyer1}{015} \\
  \ShowcaseRow{lab}{foyer1}{021} \\
  \ShowcaseRow{lab}{foyer1}{040} \\
  \ShowcaseRow{lab}{foyer2}{002} \\
\end{tabular}

\clearpage
\begin{tabular}{cccc}
  \ColHeadings \\
  \ShowcaseRow{lab}{foyer2}{006} \\
  \ShowcaseRow{lab}{foyer2}{014} \\
  \ShowcaseRow{lab}{foyer2}{017} \\
  \ShowcaseRow{lab}{foyer2}{036} \\
  \ShowcaseRow{lab}{foyer2}{041} \\
  \ShowcaseRow{lab}{foyer2}{042} \\
  \ShowcaseRow{lab}{ground1}{008} \\
\end{tabular}

\clearpage
\begin{tabular}{cccc}
  \ColHeadings \\
  \ShowcaseRow{lab}{ground1}{011} \\
  \ShowcaseRow{lab}{ground1}{020} \\
  \ShowcaseRow{lab}{ground1}{025} \\
  \ShowcaseRow{lab}{ground1}{032} \\
  \ShowcaseRow{lab}{ground1}{039} \\
  \ShowcaseRow{lab}{kitchen1}{004} \\
  \ShowcaseRow{lab}{kitchen1}{030} \\
\end{tabular}

\clearpage
\begin{tabular}{cccc}
  \ColHeadings \\
  \ShowcaseRow{lab}{kitchen1}{044} \\
  \ShowcaseRow{lab}{kitchen1}{078} \\
  \ShowcaseRowFirst{som}{corr1}{001} \\
  \ShowcaseRow{som}{corr1}{012} \\
  \ShowcaseRow{som}{corr1}{015} \\
  \ShowcaseRow{som}{corr1}{018} \\
  \ShowcaseRow{som}{corr1}{020} \\
\end{tabular}

\end{center}


\section{Payoffs}

In this section we provide some visualizations of the payoff matrices
discussed in the main paper. We hope these visualizations give extra
insights into the strenghts and weaknesses of each sensor modality.

\newcommand\FooPayoffImg[2]{
        \parbox[c]{1em}{
                \includegraphics[width=0.3\textwidth]{figures_addendum/lab_foyer2_frame#1_#2.png}}}
\newcommand\PayoffImg[1]{\FooPayoffImg{010}{#1}}

\newcolumntype{I}{>{\arraybackslash} m{.3\linewidth} }
\newcolumntype{N}{>{\arraybackslash} m{.3\linewidth} }
\newcolumntype{T}{>{\arraybackslash} m{.6\linewidth} }

%\begin{tabular}{lp{0.6\textwidth}}
\begin{tabular}{IT}
  \PayoffImg{orig} &
  The raw image provided as input to our system. \\
\end{tabular}

\begin{tabular}{IT}
  \PayoffImg{gt} &
  The ground truth segmentation for this image. Horizontal surfaces
  are shaded blue. Vertical surfaces are shaded red and green. We will
  refer to these as the ``red'' and ``green'' orientations.\\
\end{tabular}

\begin{tabular}{IT}
  \PayoffImg{dp} &
  The MAP indoor manhattan model $\Model$ output by our system for
  this input. \\
\end{tabular}

\begin{tabular}{IT}
  \PayoffImg{monopayoffs0} &
  Payoffs $\MonoPayoff$ derived from monocular image features, for the
  ``green'' orientation. Pixels of higher intensity correspond to
  larger values in the payoff matrix. The MAP model is shown in
  wireframe using red lines. Intuitively, the optimization over models
  can be thought of as finding the minimal cost path through the
  payoff matrix, where higher intensity pixels correspond to lower
  costs. This is only a rough picture, however; the real optimization
  situation is more complex since models are penalized for each
  additional corner. \\
\end{tabular}

\begin{tabular}{IT}
  \PayoffImg{monopayoffs1} &
  As above, for the ``red'' surface orientation. \\
\end{tabular}

\begin{tabular}{IIN}
  \FooPayoffImg{009}{orig} &
  \FooPayoffImg{011}{orig} &
  Auxiliary images used for stereo photoconsistency. In our
  experiments we used two image auxiliary images for each base image,
  which were sampled one second before and one second after the base
  image in the video sequence.
\end{tabular}

\begin{tabular}{IIN}
  \PayoffImg{stereopayoffs_aux0} &
  \PayoffImg{stereopayoffs_aux1} &
  Payoffs $\StereoPayoff$ corresponding to the auxiliary images
  above. Each pixel represents the photoconsistency score for a wall
  segment with floor/wall (or ceiling/wall) intersection $\ModelAtX$
  at that pixel. Notice the repeated ``pizza slice'' patterns in which
  one tip of the triangle is located at the floor/wall intersection. \\
\end{tabular}

\begin{tabular}{IT}
  \PayoffImg{points} &
  Here we show the structure--from--motion point cloud. The points are
  shown projected into the image, but the system has access to their
  3D locations. Notice how the points are not uniformly distributed in
  the image.
\end{tabular}

\begin{tabular}{IT}
  \PayoffImg{projs} &
  Here we show the structure--from--motion point cloud projected onto
  the floor and ceiling planes, which were recovered as a seperate
  step as described in the main paper. The red dots show the original
  3D point cloud and the blue dots show the projections onto the floor
  and ceiling.\\
\end{tabular}

\begin{tabular}{IT}
  \PayoffImg{3dpayoffs_agree} &
  This shows the component of the payoffs $\DepthPayoff$ intended to
  provide a bias towards models that explain the observed 3D
  points. This is the component corresponding to $t=\ON$. Each bright
  spot corresponds to the projection of a 3D point onto the floor or
  ceiling plane.
\end{tabular}

\begin{tabular}{IT}
  \PayoffImg{3dpayoffs_occl} &
  This shows the component of the payoffs $\DepthPayoff$ intended to
  penalize walls that occlude observed 3D points. This corresponds to
  the case that $t\in\{\IN,\OUT\}$. Notice that for each 3D point,
  payoffs are assigned to walls that pass between the floor and
  ceiling projection of that point. Such walls are precisely those
  which do \textit{not} occlude the point.\\
\end{tabular}

\begin{tabular}{IT}
  \PayoffImg{payoffs0} &
  Joint payoff matrix $\JointPayoff$. \\
\end{tabular}
